\documentclass{resume} % Use the custom resume.cls style

\usepackage[left=0.75in,top=1in,right=0.75in,bottom=1in]{geometry} % Document margins
\usepackage{xcolor}
\usepackage{hyperref}
\hypersetup{
    colorlinks=true,
    linkcolor=blue,
    filecolor=magenta,      
    urlcolor=teal,
}
\usepackage{graphicx}
\newcommand{\tab}[1]{\hspace{.2667\textwidth}\rlap{#1}}
\newcommand{\itab}[1]{\hspace{0em}\rlap{#1}}
\newcommand{\spazio}{\begin{center} \par\noindent\rule{0.2\textwidth}{0.4pt} \end{center}}

\name{Emilio Berti} % Your name
\address{(+45) 266 54 662 \\ emilio.berti90@gmail.com}

\begin{document}

\begin{rSection}{Education}
Since my first lab experience during my BSc, it was clear to me and others that I excelled in quantitative biology.
At the time, I was particularly intrigued by human physiology and I joined the physiology lab lead by Prof. Marco Linari, where I received a BSc degree studying the structural effects of the protein myopalladin on muscle fibers.
After this, I moved to Pavia (Italy) to enroll in one of the most prestigious neurobiology curriculum offered in Italy.
However, most of the topics I was really interested in required a strong mathematical background, which I did not have.
I quit neurobiology and returned to Florence, where I was too late to enroll in another curriculum and had to wait the next academic year.
I spent this year studying mathematics on my own and attended public courses of linear algebra and data analysis offered by the department of physics.
I did not want that the lack of mathematical skills would jeopardize again the type of research I wanted to conduct; so far, this personal training has been very helpful during my career to investigate the research questions I was interested in. Together with neurobiology, Ecology was the other course that captivated me during my BSc.
I thus enrolled the next academic year in the Ecology curriculum in Florence.
For the first time, I felt a deep connection with my studies and was extremely motivated by its intellectual challenge.
I joined the lab of Prof. Giacomo Santini, where I studied the behavior of ants during conflict.
After obtaining the MSc in Florence, I moved to Aarhus (Denmark), where I joined as a PhD student the lab of Prof. Jens-Christian Svenning, where I studied how the extinctions of megafauna caused by humans have impacted ecosystems worldwide. 
I published two peer-reviewed papers in prestigious international journals and successfully defended my PhD dissertation three years after the start of the PhD.

\spazio

{\bf PhD} \hfill {\em 01/02/2017 -- 29/06/2020} 
\\ Section of Ecoinformatics and Biodiversity
\\ Department of Biology
\\ Aarhus University
\\ Aarhus, Denmark
\\ Title of dissertation: \textit{Megafauna extinctions, allometric scaling and biotic interactions: ecological effects and restoration opportunities through rewilding}.

{\bf Visiting PhD student} \hfill {\em 15/09/2018 -- 12/12/2018}
\\ Department of Ecology and Evolution
\\ University of Chicago
\\ Chicago, IL

{\bf MSc in Biology} \hfill {\em 01/09/2013 -- 31/04/2016}
\\ Department of Ecology and Evolution
\\ University of Florence
\\ Florence, Italy
\\ Title of dissertation: \textit{Analysis of the movement and aggressive interactions between two species of ants of the Genus Lasius (Hymenoptera: Formicidae) through mathematical models}.

\clearpage

{\bf BSc in Biology} \hfill {\em 01/09/2009 -- 31/09/2012}
\\ Department of Physiology
\\ University of Florence
\\ Florence, Italy
\\ Title of dissertation: \textit{The role of the protein Myopalladin on the force generated by sarcomeres in muscle fibers}.
\end{rSection}

\begin{rSection}{Professional Experience}
{\bf PostDoctoral researcher} \hfill {\em 01/10/2020 -- Present}\\
Theory in Biodiversity Science\\
German Centre for Integrative Biodiversity Research (\textsc{iDiv})\\
Leipzig, Germany

{\bf Scientific consultant} \hfill {\em 01/05/2020 -- 31/07/2020}\\
Department of Bioscience\\
Aarhus University\\
Aarhus, Denmark

{\bf Teaching assistant} \hfill {\em 01/02/2017 -- 30/04/2020}
\\Department of Biology\\
Aarhus University\\
Aarhus, Denmark
\end{rSection}

\begin{rSection}{Research interests}
My main research interests lay at the intersection of species interactions, community assembly, and biodiversity loss due to climate change and human pressure.
I was first exposed to these topics during my MSc by Prof. Guido Chelazzi and his book \textit{L'impronta originale. Storia naturale della colpa ecologica}, which introduced me to the deep-time impacts that humans have had on ecosystems.
I broadened and deepened these interests by joining the lab of Prof. Jens-Christian Svenning at Aarhus University for my PhD.
Here, I studied the effects of human-driven extinctions on ecosystem connectivity (published in Global Ecology and Biogeography) and how human perception of species is biases towards large charismatic species (published in Biological Conservation).
During my PhD, I also investigated how the networks of trophic interactions have been altered by human-driven extinctions and how rewilding may help restoring such losses using species distribution models (published as a chapter in my dissertation).
During the last months of my PhD, I tried to secure funding for myself by applying to the Royal Society Industry Fellowship in 2020 with a project titled \textit{Evolutionary attractors of populations} that aimed at quantifying the demographic fragility of species in fluctuating environments.
Despite unsuccessful, this project exposed me to basic demographic theory and to the field of stochastic demography.

After my PhD, I was briefly employed by a research institution in Denmark, but decided to move to iDiv in Germany and pursue a PostDoc position in theoretical ecology, integrating the fields of macroecology, biogeography, and ecological networks.
	During my stay at iDiv, I focused on advancing the mechanistic understanding of animal movement (published in Methods in Ecology and Evolution and in Plos Biology), on investigating how species interactions shape community assembly at local and regional scales (published in Ecology Letters), its influence on biodiversity-ecosystem functioning (accepted in Functional Ecology as \textit{Amyntas, A., \textbf{Berti, E.}, \dots, Brose, U. -- Niche complementarity among plants and animals can alter the biodiversity-ecosystem functioning relationship}), and how these upscale to the landscape and ecosystem level (in preparation).
Moreover, I have been focusing during the last year on how inter-annual climate variability shape species' distribution and developed a novel approach that integrates stochastic demography with species' distribution modeling to account for it.
This is urgently needed to accurately projects of the spatial patterns of biodiversity will change in the future due to climate change, which is altering also the climate variability, a factors that is important in shaping species' distribution but that most SDMs currently ignore.
	I also have been in charge of releasing the new version of the open-access food-web database \href{https://idata.idiv.de/ddm/Data/ShowData/283?version=3}{GATEWAy} and collaborated in other projects to help conservation of large mammals in Africa (published in Biological Conservation) and Nepal (under review as \textit{Carter, N., \textbf{Berti, E.}, Zuckerwise, A., and Pradhan, N. -- Energetics-based connectivity mapping reveal new conservation opportunities for the endangered tiger in Nepal}), to assess the drivers of lead concentration at global scale (under review as \textit{Wei, S., \textbf{Berti, E.}, \dots, Yue, K. -- Global patterns and drivers of lead concentration in inland waters}) and to use food web theory to quantify nature contributions to people (under reivew at Trends in Ecology and Evolution as \textit{Antues, A.C., \textbf{Berti, E.}, \dots, Gauzens, B. -- Linking biodiversity, ecosystem function and nature’s contributions to people (NCP): a macroecological energy flux perspective}).
\end{rSection}

\begin{rSection}{Skills}
I have acquired a very competitive set of skills from my unique career path: I have developed an outstanding mathematical and theoretical skill set through personal training and successfully applied it to investigate macroecological and biogeographical drivers of biodiversity and species interaction networks.
I try to get the best from each experience; I have failed many times and will fail again, but this is also an opportunity to learn new things and incorporate them into my personal and professional growth.

\spazio

{\bf Languages}\\
Italian (native), English (fluent), Danish (beginner), German (beginner)\\
{\bf Programming}\\
R (expert), bash (expert), python (expert), C/C++ (advanced), javascript (mostly for Google Earth Engine, proficient), SQL (postgres flavour, advanced), GIS (expert)\\
{\bf Software} \\
Linux/GNU, QGIS, Anaconda, Jupyter Notebooks, \LaTeX, Git, GitHub\\
{\bf Methods} \\
Network analysis,
mathematical modeling,
Geographic Information Systems (GIS),
geoinformatics,
data science,
statistics,
ordination and classification,
optimization,
machine learning,
species distribution models (SDMs),
climate analyses,
environmental niche modeling,
high-performance clusters, 
automation.
\end{rSection}

\begin{rSection}{Scientific Publications}
% \vspace{4ex}\hfill{\em 2020}\\
\textbf{Berti, E.}, Rosenbaum, B., Brose, U., \& Vollrath, F. (2023). Energy landscapes direct the movement preferences of elephants. (Authorea; under review in Ecography). DOI: \url{https://doi.org/10.22541/au.168373276.62196439/v1}.\\
I was the leading author of this study, where I found that movement of elephants in Kenya is strongly determined by energetic costs of movement and availability of resources.

	Amyntas, A., \textbf{Berti, E.}, Gauzens, B., Albert, G., Yu, W., Werner, A., ... \& Brose, U. (2023). The role of niche complementarity in the strengthening of the diversity-ecosystem functioning relationship over time. DOI: \url{10.22541/au.167750685.56344133/v2}. (accepted on August 2023 in Functional Ecology).\
	I provided methodological advice and code to simulate community assembly and dynamics of trophic interaction for this study, which shows how community assembly affect biodiversity-ecosystem functions.

Dyer, A., Brose, U., \textbf{Berti, E.}, Rosenbaum, B., \& Hirt, M. (2023). Heat dissipation drives the hump-shaped scaling of animal dispersal speed with body mass. Plos Biology. DOI: \url{https://doi.org/10.1371/journal.pbio.3001820}.\\
I provided methodological advice and helped developing the mathematical model for this study, which shows that dispersal speed of animals is limited by their ability to dissipate heat produced during aerobic activity.

	Vogel, S. M., Vasudev, D., Ogutu, J. O., Taek, P., \textbf{Berti, E.}, Goswami, V. R., ... \& Svenning, J. C. (2023). Identifying sustainable coexistence potential by integrating willingness-to-coexist with habitat suitability assessments. Biological Conservation, 279, 109935. DOI: \url{https://doi.org/10.1016/j.biocon.2023.109935}.\\
	I adviced on the occupancy model and performed preliminary analyses for this study, which integrateis landscape ecology and social sciences to assess the socio-ecological factors partly limiting an effective conservation of elephants and rhinos in Northern Kenya.

Terlau, J., Brose, U., Antunes, A. C., \textbf{Berti, E.}, Boy, T., Gauzens, B., ... \& Hirt, M. R. (2022). Integrating trait-based movement into mechanistic predictions of thermal performance.\\
I provided methodological advice and performed the thermal niche analysis for this study, which predicts the thermal limits of animals by integrating species traits with metabolic theory.

Gauzens, B., \textbf{Berti, E.}, Delmas, E., \& Brose, U. (2022). ATNr: Allometric trophic models in R. bioRxiv. (Under review at Methods in Ecology and Evolution as Gauzens, B., \dots, \& \textbf{Berti E.}).\\
I co-developed the code for this R package and supervised the integration of ATN models with R.

Bauer, B.*, \textbf{Berti, E.*}, ... \& Brose, U. (2022). Biotic filtering by species’ interactions constrains food-web variability across spatial and abiotic gradients. \textit{Ecology letters}. DOI: \href{https://doi.org/10.1111/ele.13995}{10.1111/ele.13995}. (\texttt{Shared first authorship}).\\
I shared first authorship of this paper, where we showed that species composition is determined by the interplay between environmental factors and species interactions using a global food-web database.

Grenié, M, \textbf{Berti, E.}, ... \& Marten, W. (2022). Harmonizing taxon names in biodiversity data: a review of tools, databases, and best practices. \textit{Methods in Ecology and Evolution}. DOI: \href{https://doi.org/10.1111/2041-210X.13802}{10.1111/2041-210X.13802}.\\
I co-developed the conceptual framework of this paper, where we identified current gaps in how ecological datasets are integrated for synthesis projects and suggested potential solutions. I also performed the analyses for the case study presented.

\textbf{Berti, E.}, Davoli, M., ... \& Vollrath, F. (2021). The r package enerscape: A general energy landscape framework for terrestrial movement ecology. \textit{Methods in Ecology and Evolution}. DOI: \href{https://doi.org/10.1111/2041-210X.13734}{10.1111/2041-210X.13734}.\\
I developed the R package and lead this study, where I integrate energy landscapes and GIS tools in order to provide macroecologists and conservation scientists with a null-model of animal movement based only on energetic costs of movement.\\
This paper received also a press release that drew attention of several media outlets, most notably the German national newspaper \textit{Die Welt} and the public radio broadcast \textit{Mitteldeutscher Rundfunk}, operating in the Federal States of Thuringia, Saxony and Saxony-Anhalt.

\clearpage

\textbf{Berti, E.}, Monsarrat, S., Munk, M., Jarvie, S. \& Svenning, J.C. (2020). Body size is a good proxy for vertebrate charisma. \textit{Biological Conservation}. DOI: \href{https://doi.org/10.1016/j.biocon.2020.108790}{10.1016/j.biocon.2020.108790}.\\
I lead and performed analyses of this study, where I show that humans are more positively attracted to animals, potentially biasing funding and conservation efforts towards protection of large animals, rather than species that are more critically endangered.

\textbf{Berti, E.} \& Svenning, J.C. (2020). Megafauna extinctions have reduced biotic connectivity worldwide. \textit{Global Ecology and Biogeography}. DOI: \href{https://doi.org/10.1111/geb.13182}{10.1111/geb.13182}.\\
I lead and performed analyses of this study, where I show how human-driven extinctions in the last 50,000 years have limited the ecosystem processes depending on animal dispersal.
\end{rSection}

\begin{rSection}{Peer review}
	As of August 2023, I have reviewed 7 papers for: Ecography (2), Ecology Letters (2), GigaScience (1), Methods in Ecology and Evolution (1), and Scientia Agricola (1). You can find more at my \href{https://www.webofscience.com/wos/author/record/2190178}{WoS profile}.
\end{rSection}

\begin{rSection}{Conference talks}
\underline{Invited talk}: Bauer, B., \textbf{Berti, E.}, \dots, \& Brose, U. (2022). From regional to local scale: biotic interactions shape multilayer food-webs. \textit{SFE-GFO-EEF biannual meeting, Metz, France}.

\textbf{Berti, E.}, \& Svenning, J.C. (2022). State-space models show that functional replacements of extinct megafauna have distinct habitat preference in a European rewilding area. \textit{SFE-GFO-EEF biannual meeting, Metz, France}

Grenié, M., \textbf{Berti, E.}, Carvajal-Quintero, J., Winter, M., \& Sagouis (2021). Matching Species Names Across Biodiversity Databases: Sources, tools, pitfalls and best practices for taxonomic harmonization. \textit{TDWG annual meeting, online}

\textbf{Berti, E.} \& Svenning, J.C. (2019). Megalinkers extinction and the decrease of ecosystem connectivity. \textit{ESA annual meeting, Louisville, KY}

\textbf{Berti, E.}, Jarvie, S. W., \& Svenning, J.C. (2018). Rewiring food webs via trophic rewilding. \textit{BES annual meeting, Belfast, UK}
\end{rSection}

\begin{rSection}{Supervision and Mentoring}
I am informally co-supervising two PhD students at iDiv: Angelos Amyntas, whose work focuses on biodiversity-ecosystem functioning, and Ana Carolina Antunes, whose work focuses on the impacts of humans on $\alpha$-diversity using camera trap data. I am formally co-supervisor of the PhD candidate Jingyi Li, who is developing a novel mathematical approach for functional responses based on information theory. In addition to supervision, I also provide theoretical, computational, and statistical advice to several members of the TiBS working group at iDiv as well as individual mentoring for PhD students, advising especially on transferable skills and alternative career paths outside academia.
\end{rSection}

\begin{rSection}{Teaching \& Organized Workshops}
During my PhD, I taught one course every semester, as part of my salary was paid by Aarhus University on the basis of teaching.
I also teach in guest lectures in my current PostDoc role and regularly organize lab seminars and workshops to teach reproducible research and open data principles.
I have thus a relatively long teaching experience for my career stage.

\spazio

{\bf Teaching} \\
Introduction to scientific programming and tidyverse (2022) -- \href{https://emilio-berti.github.io/teaching/tidyverse.html#(1)}{slides}\\
Introduction to git and GitHub for a fool-proof programming (2022) -- \href{https://emilio-berti.github.io/idiv-git-introduction/}{course}\\
{\bf Teaching Assistant} \\
Theoretical Population Ecology (2023)\\
Meta-analyses for Biodiversity (2021)\\
Statistical and Geospatial Modelling (2019)\\
Behavioural Biology (2018, 2019)\\
Geographic Information System (2017)\\
{\bf Organized Workshops}\\
Cleaning online repository data for use in biogeography and macroecology (2019)\\
Running a species distribution model in R (2019)\\
A (very) gentle introduction to Linux (2019)
\end{rSection}

\begin{rSection}{Collaborations}
I have moved several times to pursue my career dreams and, being a friendly person, I established personal and professional ties with the people I met.
My network of current collaborators include:
\begin{itemize}
\setlength\itemsep{-0.5em}
    \item Prof. Jens-Christian Svenning, Aarhus University, Aarhus, Denmark.
    \item Prof. Ulrich Brose, Jena University, Jena, Germany.
    \item Prof. Daniel Reuman, Kansas University, Lawrence, KS, USA.
    \item Prof. Giacomo Santini, Universit\`{a} degli Studi di Firenze, Florence, Italy.
    \item Prof. Kai Yue, Fujian Normal University, Fuzhou, China.
    \item Prof. Neil Carter, University of Michigan, Ann Arbor, MI, USA.
    \item Ass. Prof. Susanne Vogel, Open University of the Netherlands, Heerlen, Netherlands.
    \item Prof. Fritz Vollrath, University of Oxford, Oxford, UK.
\end{itemize}
I also collaborate with Dr. Sophie Monsarrat, rewilding manager at the NGO Rewilding Europe (Nijmegen, Netherlands), and veterinary doctor Agnese Santi (Prato, Italy), with whom I am developing a new concept of ``wildness'' that can be applied to feral and semi-domesticated horses in Europe and that can lead to the identification of existing horse populations that promote biodiversity and ecosystem services.
\end{rSection}

\begin{rSection}{External Links}
\begin{minipage}{0.5\textwidth}
\begin{itemize}
    \item \href{https://scholar.google.com/citations?user=5KPh-oUAAAAJ&hl=en}{Google Scholar profile}
    \item \href{https://emilio-berti.github.io/}{Personal website}
    \item \href{https://orcid.org/0000-0001-9286-011X}{ORCiD}
\end{itemize}
\end{minipage}
\begin{minipage}{0.5\textwidth}
\begin{itemize}
    \item \href{https://www.linkedin.com/in/emilio-berti-55a348146}{LinkedIn}
    \item \href{https://github.com/emilio-berti}{GitHub}
    \item \href{https://publons.com/wos-op/researcher/4208953/emilio-berti/peer-review/}{Publons}
\end{itemize}
\end{minipage}
\end{rSection}

\end{document}
